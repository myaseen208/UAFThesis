\chapter{Review of Literature}\label{review_chapter}
The information on the topic is not inadequate but the inference of different research are not consistent and exhibit  differences reported to the  material used and location of experiment . However, some studies from literature are reviewed here which are relevant to the subject.  

\section{Classical approach} 
A research was conducted to study  different stability measures from AMMI model for the analysis of stability. For This purpose a data of two year (2004-2005), which consist of 17 genotypes of chickpea  across five environments was take. The experiment was Randomized complete block design with 4 replications. The relationship stability measure was assessed using Spearman's rank correlation.The analysis also shown that stability measures can be categorized into three groups. The first group encompassed, $SIPC_4$,$ EV_4$, MASV, $Dz_4$,$ AV _{(AMGE)}$,$ W_{i(AMMI)}$ and $FA_i$ on the basis of correlation with mean yield. ASV, $Da_4$, $B_i$ and$ Za_4$ were put in 2nd group for similar criteria. Group 3 contain of $FP_i$ that was not related with yield. Then Principal component analysis was performed to understand relationship among different methods. Bi-plot of PC 1 and PC 2 was made, analysis showed that 84\% variation of GE interaction can be explained by first four multiplicative terms. On the basis of high mean  yield and according to all measures, it was recommended that genotype G13 (FLIP 97-114) can be adapted widely across environments for good results \citep{Zali2012}.\\ 
 
For analysis of multi-environment data \citep{Gauch2006} made comparison of merits between two theories which purely base on singular value decomposition. The models were AMMI and genotype main effect and genotype by environment interaction (GGE), principal component analysis was also performed. It was observed that as far as model choice concern, AMMI was best among them for agriculture analysis because over all variation can be partitioned into three sources by it. where as for predictive accuracy, all methods proved to be equivalently efficient. But it was suggested that to draw useful inference from the data all required model diagnosis. \\

 Genotypes of different genetic structure behave differently, when place in different growing conditions. The different growing responses with respect to environment are called GE (Genotype by Environment). GE interaction is a structure in which the main effects and non-additive interaction of plants breeding or genotypes are studied. There are several techniques; and ANOVA (Analysis Of Variance) is one of such technique to analyze, but for GE the interaction from additive model the effects are non-additive. So AMMI model was proposed for such two way data, AMMI (bi-plots) were used to identify the relationship of interaction. It provided clustering of genotypes on the basis of similar of outputs and identifying trends of genotypes across environment. It was also a useful technique to recognize when GE interaction is provided by insignificant genotype and environment main effects, or where the structure of the interaction is influenced by outliers.  Tai's stability statistics may also be used, along with bi-plots analysis, to measure the relative stability, reliability, and ordering of genotypes in a specific regional trial \citep{Shafii1998}.\\

 An experiment was conducted to makes comparison among three different methods SREG, regression analysis and AMMI to explore the G $\times$ E. The objective of the study was to the assessment of genotypes which show maximum yield potential and the stability in diverse environments. For this purpose an experiment in which 13 genotypes of wheat were sown in six different locations was conducted in Chile. Analysis revealed that "Pandora-INIA" gave the maximum yield in all environment and was referred as the most stable genotype, it also maintain the quality level. It was also suggest although all method were sufficiently explain the G $\times$ E, but among theses these techniques SREG was most appropriate \citep{Castillo2012}. \\

To examine the yield stability, adaptability of environments and for analyzing G$\times$E of tobacco, An RCBD experiment was evaluated which comprises 15 hybrids in 8 different environments with three replication for 2006 and 2007. AMMI analysis showed that 88\% of sum of squares was explained by the environments. Environment with large sum of squares were indication that they were diverse. From AMMI results it was also noticed that IPCA 1 explained 82\% interaction sum of square. Similarly first two IPCA cumulatively explained 91\% sum of square for interaction. Results disclosed that hybrids PVH03, K394/NC89 and Coker254/NC89 having smallest interaction, and hybrids ULT109, NC291, Coker254/Coker347 and VE1/Coker347 having highest interaction were  found the most consistent and inconsistent hybrids, respectively. Furthermore,  Rash for Coker254/K394, NC291 and CC27 non-in drought stress condition were more appropriate and for Rasht, hybrids NC89/Coker347, K394/ Coker347, Coker254/VE1 and ULT109 were more preferable in drought stress condition \citep{Sadeghi2011}.\\ 

 The objective of study was to assess whether the presence of GE affects the lint yield and fiber strength or not. For this purpose An MET data which consist of 8 genotypes of cotton in twelve different environments from South Carolina was taken and AMMI analysis was performed. Analysis revealed that GE affects the lint yield and fiber strength. 71\% variation of GE for lint yield and 70\% variation for fiber strength were explained by first two principal components. Further diagnosis showed that two mega environments in South Carolina cotton areas can be made for lint yield that is; north east areas and southern areas. On the other hand  single mega-environment can be representative for whole South Carolina region to test fiber strength and genotypes which can be widely adapted \cite{Campbell2005}. \\

A comparison was made between AMMI and joint regression to explain which method is sufficient to describe genotype-location (GL) effects along with principal component axes's variance which found significant. To asses Repeatability of stability in genotypes in different environments was also main purpose. For analysis different methods which include Euclidean distance about origin for significant PC axes (D), absolute PC values (|PC 1|) and Finlay and Wilkinson method for joint regression were used. Shukla's stability variance was also implied. This study based on three data sets including  two for maize,  and one for bread wheat and one for  oat. Results revealed that AMMI analysis was more adequate and valuable for explaining the GE interaction in six data sets. D method was proved to be more repeatable than | PC1 | and Shukla's method. As for as ordination of GL effects concerned both methods were found appropriate. It was shown that wheat and maize data set were consistent in ordination for GL interaction PC1, when neither season nor genotypes in common \cite{Annicchiarico1997}.\\

To assess genotypes which perform well in diverse environment and gave high yield, AMMI model was used by \citep{Akter2014} in Bangladesh for rice crop. The experiment layout was RCBD, which consist of 20 genotypes across 5 different places . It was explored the main effects and interaction terms were found significant. AMMI bi-plot between first two PC shown that BRRI 10A/BRRI 10R (G3) was the most stable genotype as it gave maximum yield and BRRI dhan39 (G 12)  gave the lowest yield, so it was referred as unstable genotype . It was observed that the variety G3 exhibited high yield in all environments and perform better than other genotypes. It was noticed that G$\times$E did not affect genotypes BRRI 1A/ BRRI 827R (G1), IR58025A/ BRRI 10R (G2),BRRI 10A/BRRI 10R (G3) and BRRI hybrid dhan1( (G4), so these genotypes would be stable in different environments. As far as environments were concerned, Gazipur (E1) and Jessore (E5) gave near zero IPCA scores, this mean these location can be considered  stable  for the improvement of yield rice crop. \\
 
 The aim of study was to observe stability and genotype environment interaction. For this purpose a data set of durum wheat was taken from south east Ethopia for years (2003-2005), which comprises twenty genotypes across 15 environments. The relationship among genotypes and the stable genotypes were identified using different stability parameters. Combined ANOVA and AMMI analysis was performed. In ANOVA except genotype-year interaction  other effects were observerd significant, whereas in AMMI analysis 4 multiplicative term were significant. Genotype 3 and 4 was recommended across environments as these were identified most stable genotypes. Among different stability measures $Sd_i^2$, $W_i$, $Sx_i^2$ and ASV showed high rank correlation. So it was recommended that all are equally beneficial to asses the stable genotypes \citep{Lett2007}. \\
 
   A data set which consist of  10 genotype across seven different environment in Pakistan for two consecutive years 2007-2008 was analyzed by \citep{Mujahid2011}. Analysis of variance was performed an it was observed that approximately 79\% Sum of squares(SS) was explained by environments, while genotypes explained 3\% SS and 9\% was by GE interaction. To assess grouping among environments and among genotypes Cluster analysis was applied. It was shown that environments merge in 4 groups and genotypes in 4 groups. GGE biplot between PC 1 and PC 2 was drawn, Genotype NR-314 and genotype NR-310 behave differently in environments than other genotypes. GE biplot revealed that NR-306 and NR-305 were the most stable genotypes because they gave maximum yield. \\
   
  To assess the plant production it is important to know how genotypes and environment interact with each other. A trail consist of 21 genotypes on 7 different locations of rye-grass was use to highlight the interaction. Many models were presented by statisticians but a bi-additive model fit well. Bi-plots were used to demonstrate the performances of genotypes in different locations, it was observed that genotype 1 (G1) and location 4 (L4) showed a maximum negative interaction, while genotype (G1) and genotype (G7) were far from origin so they showed a maximum positive interaction; similarly genotype (G2), genotype (G4) and genotype (G9) were poor at location 7 (L7). It was suggested that additive model are better but some areas need some attentions like, Additive model may be considered as extension of generalized linear model, how well is interpretation by confidence region and how precise and valid asymptotic formula is \citep{Denis1996}. \\ 
 
  To evaluate GE interaction and determine genotypes stability an experiment was conducted in Ethiopia (Africa) during 2004-2005. Twenty genotypes of wheat were tested in 6 different research stations. The interaction for genotypes and environment was analyzed by using AMMI and GGE biplots. Combined ANOVA showed that main effects as well as interaction effect were highly significant. GGE biplot analysis revealed that for first year 70\% SS and for 2nd year 80\% sum of square were explained by PC1 and PC2. Mega-environments were identified for first year by "which-one-where" pattern. It was suggested that a repeatability of which-won-where arrangement over years is the essential and necessary condition for mega-environment delineations and to make recommendations for genotypes \citep{Negash2013}. \\ 
  
  Environment effects on genotypes of wheat were assessed by making comparison between AMMI and GGE bi-plot. For analysis of GEI in bread wheat; an experiment was conducted in 2010/2011, and 2011/2012 seasons at the Research Farm of Faculty of Agriculture, Sohag University, Egypt. Ten wheat cultivars differing in to adapt heat in 12 environments were used. The experiment was completely randomized block design which consist of three replicates. It was observed that AMMI and GGE bi-plot models were successful in assessing the performance of genotypes and the choice of best genotypes was almost similar in both of them. On the basis of two analysis AMMI and GGE bi-plot models, G10 (Giza 168), G2 (Sakha 8) and G6 (Sids 1) were identified by high yield and were stable; therefore the G10 (Giza 168) can be considered as an ideal genotype. When this technique does not explain the interaction in grain yield, then another technique called mixed model may produce better results \citep{Mohamed2013}. \\
 
  To assess the performance of genotypes across diverse environment; an experiment was conducted at different locations in Pakistan to check the stability parameter for yield of grains. Twenty genotypes of spring wheat were evaluated in thirty one locations of Pakistan for the year 2001-02. ANOVA revealed that 98.6\% variation for GE interaction was accounted by genotypes. For stability parameter checking they used two methods; Safety-First Rules and cluster analysis. On the basis of Safety First Rules, genotypes such as V-97046, 97B2210 V-98059 and V-97052 were noted to be as stable genotypes. Moreover, the genotypes performing similar response pattern over the environments and vice versa were grouped by using cluster analysis on wheat GE interaction data. twenty genotypes were clustered into 10 groups, whereas 31 environments were merge into 16 clusters. It was concluded that the Safety-first rule is the best technique due to the reason that this technique explicitly weighs the importance of stability relative to yield, whereas cluster analysis technique is useful to asses that which genotype performed well at specific environment \citep{Rasul2007}. \\ 
 
Multi-environmental trails (MET) were graphically analyzed by using GGE bi-plot and genotype$\times$ traits bi-plot commonly. A new bi-plot technique covariate-effect bi-plot was used on MET data set of barley. For this purpose an experiment was piloted in North America, which was consisted of one hundred forty five genotypes across 25 environments. A comparison among all methods was done, GGE and covariate effect bi-plot explored that environment can be divided into two meg-environments. Furthermore, 81\% pattern of GGE was explained by covariate effect bi-plot. This suggested that indirect selection for trait on basis of yield can effectively reconnoitered by the GGE pattern. In particular for east environments, selection of large kernel weight, good loading resistance, initial heading can be used to improve the yield of barley. While for western environments, yield can be improve only by the selection of yield per se through environments. It was suggested that by using all methods jointly, MET can be analyzed in much better way \citep{Yan2005}. \\

 To assess the stability and adaptability an experiment was conducted by \citep{KAYA2002} in Turkey during year (2000-2001).  The experiment was consist of 20 genotypes of wheat crop in six different environments with four replication. The layout of experiment was Randomized Complete Block Design. First, Combined ANOVA was performed for six environments and all effect were found significant. Secondly, AMMI Analysis was performed and it was observed that environmental impact has significant effect on the production of Wheat crop. 100\%  interaction was explain by first five principal components, whereas PC 1 and PC 2 almost explain 78\% interaction. \\ 
 
To examine Stability performance an experiment was evaluated for thirty genotypes of wheat in six different stations. The pattern of experiment was RCBD with three replications.  AMMI and GGE analysis was performed for the evaluation of genotypes and analysis explored that G$\times$E is highly significant. AMMI stability value (ASV) showed that 14 (Irena $\times$ Veery) have high mean yield so referred as most stable.  The GGE revealed that crosses number 11 (Irena $\times$ Chamran) and 17 (S-78-11 $\times$ Chamran) were the most stable combination, and it was recommended these can be used for  the making hybrids \citep{Rad2013}. \\
 
 To study the GE interaction and stability measure, an experiment in Ethiopia was conducted during  2007 and 2008.  At sixteen different environments 14 genotypes of pea were appraised, the trail was carried out in RCBD layout with 4 replicates. AMMI analysis and site regression(SREG) bi-plot method were applied for assessment, pooled ANOVA revealed that main effects and interaction effects were significant, two component explain 69\% sum of squares for interaction with 52 degree of freedom. The  initial  five bilinear terms were observed important in AMMI. Except EH02-036-2 and COll.026/01-4 genotypes no genotype showed better performance than others, as these exhibit top ranking among five out of 16 environment. It was validated that both method can effectively be used for visual comparison and to identify the superior genotypes. It was suggested that indirect selection of environment can be proved effective for the identification of better genotype performance \citep{Tolessa2013}.\\
 
 To assess that G$\times$E plays an important role in pasta color which an important trait, an experiment was conducted which included 18 genotypes of wheat crop sown in 13 different sites. Main effects and interaction effects were observed highly significant when combined ANOVA was performed. Ranking changes in genotypes did not show any sign of significance. Among all genotypes, G11 adapted the conditions well enough because it gave maximum grain yield, it was also indicated that the pasta color potential can be improved of semolina. Furthermore, for grain yield as compared to non-crossover type, cross-over type was found more important. Similar pattern was observed for pasta color of semolina.A particular local adaptation arrangement was observed as high GY, TW and semolina yellowness, was identified, also less correlation among these will facilitation in improve of pasta color without effecting the production and quality.  \citep{Schulthessa2013}. \\
 
 Although AMMI model can analyzed G$\times$ E adequately which is based on singular value decomposition (SVD), but problem arise when there are extreme values or outliers which can make data contaminated. As AMMI use least square method can be significantly affected by these contaminations because OLS is sensitive to outliers. \citep{Rodrigues2015} proposed a robust AMMI (R-AMMI) model to tackle these fragility of classical AMMI model. A simulated as well as two real data sets was used for analysis. It was observed that in classical AMMI OR91 exhibit a significant effect on biplot and shown overlapping in different direction, whereas, R-AMMI biplot despite all influence of OR91 displayed a better visual and made interpretation rather easy. Results explored that R-AMMI can be used to obtain successive principal components, Moreover, similar result and interpretation can be applies on R-AMMI biplots. It was also suggested that precautionary measure should be taken while cleaning detectable measurements.  \\
 
 Treatments and Blocks are two factors for randomized complete block Design (RCBD); if treatments are fixed, best linear unbiased estimation (BLUE) method is better, if treatments are random, best linear unbiased prediction (BLUP) method is preferable because it reduces the treatment means provides less root-mean-square error (RMSE). Practically the variance components are estimated. Prediction obtained through estimated variance component, is called empirical best linear unbiased prediction (EBLUP), but EBULP cannot be reliable when the experiment is small. A simulation was used to assess performance of EBLUP with normally and non-normal random effects and was compared with Bayesian approach. It was observed that EBLUP performance was better as compared to BLUE for RMSE, as well as for non-normally distributed treatment. The Bayesian method provided the smallest RMSE and more precise prediction intervals than other methods \citep{Forkman2013}.\\
 
\section{Bayesian approach}
Bayesian inference is more useful because it provides easy interpretation of statistical conclusions. The output of analysis is based on posterior distribution, so for unknown parameters it gives the ability to estimate intervals. So this property of Bayesian inference provides flexibility to fit any model for multi parameters \citep{Gelman2004}.\\

 Genotype rank changes across environments, were compared which are termed as crossover interaction (COI). A comparison was made between two bilinear models, the sites regression model (SREG) and shifted multiplicative model (SHMM). Two cultivar, one comprised of 20 genotypes of maize evaluated in fourteen international locations, layout of trail was RCBD with 4 replications. Other data consist of sixty genotypes of wheat in 5 distinguished sites with 2 replication in each site, trail was designed on RCBD layout. For maize dataset, cluster analysis on non-COI for environments was done, whereas wheat data was used for clustering genotypes.  It was observed that these methods were valid for clustering different location and genotypes on non-crossover GEI subsets \citep{Crossa2004}. \\
  
  Estimates of multiplicative interaction can be obtained by Bayesian approach which uses Gibbs sampling with embedded Metropolis-Hasting random walks \citep{Viele2000}.\\
 
 Principal component analysis (PCA) is dimensions reduction technique of models by rotation of axes. Different noise can be accounted by using an extension of PCA called Bayesian Principal Component Analysis (BPCA). But prior information cannot be utilized by PCA or its extensions. It was showed that BPCA not only estimate the parameters precisely, but also take measurements in much better way. The BPCA algorithm assume that the rank of model is known or it can be estimated, and that the noise follows the Gaussian distribution, but BPCA method is useful even if noise is not Gaussian. Furthermore, BPCA showed more robustness for errors to estimate the rank of model. The proposed BPCA is useful for to tackle PCA or MLPCA problem such as to estimate prior distribution by using Monte Carlo methods \citep{Nounou2002}.\\ 

 The problem with the least square estimates of AMMI model was, it did not collaborate the GE interaction of first two components into bi-plots. To over-come this difficulty an alternative Bayesian approach was suggested. For this purpose, a study was carried out using a data set of grain yield, it includes nine genotypes assessed in 20 locations using a RCBD with 4 replications, vague but proper prior was applied. Bayesian bi-plot of bilinear terms was plotted and observed that genotype 1 and 8 perform significantly different from other genotypes and form a group with negative values. Genotype 4, 5, 6, and 9 were on right side and merged into another definite group of genotypes having positive values. Highest posterior density (HPD) interval of 95\% and 99\% probabilities were formed, and those which did not include null point (0, 0) were referred as highly significant. It was suggested that this new method is better  way to assess the bilinear terms which remained under-shadow in other methods \citep{Crossa2011}. \\
 

  In plant trial AMMI models are widely used to explore GEI, but criteria for the selection of number of multiplicative terms should retain in the model, which can sufficiently explain GEI is debatable issue with AMMI. For this purpose a study was conducted which comprises evaluation of 55 genotypes across nine environments. Randomized complete block design layout was used with three replicates.A comparison between Bayesian AMMI and Bayesian shrinkage AMMI was made; both methods show advantages over one another. Bayesian AMMI gave low shrinkage singular values as compared to Bayesian shrinkage AMMI , but gave useful path for the determination of credible interval. On the other hand, Bayesian Shrinkage AMMI gave principal components which have high shrinkage value as compared to mixed models. It allowed the selection of model vary much same as classical AMMI model. It was suggested that Bayesian shrinkage AMMI can be used for estimation of credible intervals without taking  Gaussian assumptions into account which are required in classical methods \citep{Silva2015}.\\ 
  
  Bayesian approach was applied for linear bilinear models inference by using a prior distribution called Von Mises-Fisher distribution. Instead of using orthogonal eigenvectors, orthogonal matrices were used on MCMC samples. Bayes factor was used to observe that which bilinear terms were significant. For analysis; first a simulated data was used. It was consist of 5 rows and 3 columns with 4 replication, for this posterior distribution was estimated and bi-plots were constructed. From the summary of marginal posterior distribution, HPD interval (90\% and 95\%) indicated that bilinear terms which do not include point (0,0) were significant. It was observe that eigenvectors for $u_{1,1}$, $v_{1,1}$,$ v_{2,2}$ and $v_{3,2}$ were the significant. The significance was also shown by using bi-plots that genotype 1 and environments $S_1$, $S_2$ and $S_3$ were significant. 
  
  Further, about the clusters among rows and columns were inspected using dendrogram. Then, a real data set of 12 maize hybrids and of 25 environment for consecutive years was used; Where, first year data were used as prior. Histogram of MCMC samples of marginal posterior and for cumulative proportion of variance explained were shown and indicates that 5 component explain approximately 90\% interaction variance. Bayes factor 104.3 indicated that 3 bilinear components are appropriate. Bi-plots indicated that genotype 12 was significant, and environments such as S5, S9, S23 and S25 were significant for "GE" interaction variability, joint description was also given; dendrogram of hierarchical clustering algorithm was also shown \citep{PEREZ-ELIZALDE2011}.\\
 
 In AMMI models the problem of over parameterization arises (there is over parameterization when alteration of any non-constant function of the parameters do not alter the likelihood). It was suggested that this problem can be tackled by defining directly priors on all the parameters without taking into account the constraints by applying an appropriate post processing at the posterior level. This problem can be tractable by standard MCMC algorithm. So genotype environment dataset of 16 Genotypes was used and there were some question needed to be answered, like general performance of genotype, Specific performance of genotype and risk involved with genotype.
 
  Firstly, from the graphical representation it was shown that Genotypes 1-6, have  positive main effects, Genotypes 8-14 have negative main effect and Genotypes 15-16 were stable. Secondly, behavior of Genotypes in different environment, It was observed that 1-6 genotypes performed better in poorest environments, 8-14 perform good behavior in the best environment, whereas 15-16 observation were stable in all conditions. Finally it was shown that the risk related to the Genotype that yield will be lower than a specific level, It was shown that Genotype 3 and 16 have probabilities of .16 and .19 have yield less than 4. It was suggested that Bayesian approach can answer these question in much better way. Further, to use effects either fixed or Random can be differentiated by using different prior \citep{JOSSE2014}.